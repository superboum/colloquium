\documentclass[]{article}
\usepackage[T1]{fontenc}
\usepackage{lmodern}
\usepackage{amssymb,amsmath}
\usepackage{ifxetex,ifluatex}
\usepackage{fixltx2e} % provides \textsubscript
% use upquote if available, for straight quotes in verbatim environments
\IfFileExists{upquote.sty}{\usepackage{upquote}}{}
\ifnum 0\ifxetex 1\fi\ifluatex 1\fi=0 % if pdftex
  \usepackage[utf8]{inputenc}
\else % if luatex or xelatex
  \ifxetex
    \usepackage{mathspec}
    \usepackage{xltxtra,xunicode}
  \else
    \usepackage{fontspec}
  \fi
  \defaultfontfeatures{Mapping=tex-text,Scale=MatchLowercase}
  \newcommand{\euro}{€}
\fi
% use microtype if available
\IfFileExists{microtype.sty}{\usepackage{microtype}}{}
\ifxetex
  \usepackage[setpagesize=false, % page size defined by xetex
              unicode=false, % unicode breaks when used with xetex
              xetex]{hyperref}
\else
  \usepackage[unicode=true]{hyperref}
\fi
\hypersetup{breaklinks=true,
            bookmarks=true,
            pdfauthor={},
            pdftitle={},
            colorlinks=true,
            citecolor=blue,
            urlcolor=blue,
            linkcolor=magenta,
            pdfborder={0 0 0}}
\urlstyle{same}  % don't use monospace font for urls
\setlength{\parindent}{0pt}
\setlength{\parskip}{6pt plus 2pt minus 1pt}
\setlength{\emergencystretch}{3em}  % prevent overfull lines
\setcounter{secnumdepth}{0}

\author{}
\date{}

\begin{document}

\section{Cahier des Charges:}\label{cahier-des-charges}

\subsection{Contexte}\label{contexte}

Le Centre de Mathématiques de l'INSA de Rennes organise une conférence
de Mathématiques en juillet 2016, qui durera une semaine. 150
participants sont attendus.\\Il est donc nécessaire de créer un site
web, à mettre en ligne en juin 2015 afin de promouvoir et d'organiser la
conférence.\\Spécifications -------------- Le nom de la conférence,
\textbf{IWSM} devra être visible.\\Le site devra présenter une partie
publique, et une partie réservée à l'administration, non accessible par
le grand public.\\La partie publique comportera plusieurs onglets,
correspondant à plusieurs fonctions. Celles ci sont listées ci dessous
:\\* Une partie administrative, pour la gestion des inscrit à la
conférence. On pourra : * s'inscrire* * soumettre et consulter un
résumé\emph{ } demander un soutien financier\\ Les deadlines seront
spécifiées. * Des renseignements pratique (comment venir à Rennes,
logement, \ldots{} ) * Contact\\* Sponsors L'administrateur doit avoir
accès à toutes les parties citées ci-dessus. Cependant, il devra
également pouvoir : * Accéder aux statistiques ( nombre d'inscrits,
rapport ,\ldots{} ) * Le site devra être personnalisable*.

Les mentions suivies d'une étoile seront décrite plus amplement par la
suite. Inscription : -------------

L'inscription sera entièrement gérée par le site, et donc
automatique.\\Les différents champs suivants devront être renseignés : *
Nom et prénom * Adresse mail * Établissement d'origine * Durée du séjour
* Nombre de repas * Inscription aux excursions proposées pendant la
relâche. * contraintes nourriture (végétarien, religion, \ldots{} )

Toute inscription générera l'impression d'une carte de visite
personnalisée au nom du nouvel inscrit.\\De même, un prix sera
automatiquement calculé, suivant les informations qui auront étés
données ci dessus.

\begin{enumerate}
\def\labelenumi{\arabic{enumi}.}
\itemsep1pt\parskip0pt\parsep0pt
\item
  Le site générera une facture qui sera payable via bon de commande. Un
  administrateur pourra alors attester de la bonne réception du cheque
  ou bon de commande et valider le paiement.
\item
  En seconde partie il pourra être envisagé un paiement en
  ligne.\\Toutes les inscriptions seront enregistrées dans une base de
  données, restituable à l'administrateur sous forme d'un fichier Excel
  par exemple. Celui ci constituera un récapitulatif complet de chaque
  inscrit.
\end{enumerate}

\subsection{Résumés :}\label{ruxe9sumuxe9s}

La soumission de résumé sera faite en ligne. On pourra utiliser l'outil
ConfTool pour créer cette partie.\\Un résumé est un document envoyé à la
suite de la conférence par un inscrit. Celui ci doit être relu par un ou
plusieurs membres. Par conséquent, il devra donc être possible de : *
soumettre un résumé, pour un membre ayant assisté aux conférence *
modifier un résumé préalablement soumis * soummetre un résumé final *
Gérer la relecture des résumés: * L'administrateur désigne un ou des
relecteurs. * Les relecteurs doivent pouvoir être averti facilement par
mail. * les résumes sont téléchargeable via le site web, pour les
personnes autorisées. * Le relecteur doit pouvoir spécifier si le résumé
est accepté, à modifier ou rejeté. Un commentaire peut accompagner
l'avis. * L'auteur du résumé est prévenu par mail de la modification de
l'état du résumé (notification).

\subsection{Personnalisation :}\label{personnalisation}

Un administrateur du site pourra personnaliser le site. Il pourra : *
Ecrire des articles, qui seront référencés dans un ``blog'' * Créer ou
modifier une page * Créer des événements

Ces modification devront être faisable par une personne non expérimentée
en informatique.

\subsection{Langues}\label{langues}

Le site sera entièrement en anglais.\\Une traduction en allemand sera
éventuellement proposée, selon la difficulté et le temps restant. Elle
sera soit partielle ( uniquement sur certaines pages, définies par
l'administrateur ) soit complète ( Chaque modification du site devra
ainsi être faite pour les deux langues. Si la traduction allemande n'est
pas spécifié, la partie apparaîtra en anglais).

\subsection{Design}\label{design}

Le site devra être ergonomique.\\Il sera par exemple proposé une version
adaptative selon la taille de l'écran, pour qu'il puisse par exemple
être lu sur tablette.

\end{document}
