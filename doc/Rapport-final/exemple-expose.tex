%%%%%%%%%%%%%%%%%%%%%%%%%%%%%%%%%%%%%%%%%%%%%%%%%%%%%%%%%%%%%%%%%%%%%%%%%%%%%%%
%     STYLE POUR LES EXPOS�S TECHNIQUES 
%         3e ann�e INSA de Rennes
%
%             NE PAS MODIFIER
%%%%%%%%%%%%%%%%%%%%%%%%%%%%%%%%%%%%%%%%%%%%%%%%%%%%%%%%%%%%%%%%%%%%%%%%%%%%%%%

\documentclass[a4paper,11pt]{article}

\usepackage{exptech}       % Fichier (./exptech.sty) contenant les styles pour 
                           % l'expose technique (ne pas le modifier)

%\linespread{1,6}          % Pour une version destin�e � un relecteur,
                           % d�commenter cette commande (double interligne) 
                           
% UTILISEZ SPELL (correcteur orthographique) � acc�s simplifi� depuis XEmacs

%%%%%%%%%%%%%%%%%%%%%%%%%%%%%%%%%%%%%%%%%%%%%%%%%%%%%%%%%%%%%%%%%%%%%%%%%%%%%%%

\title{ \textbf{Rapport d'�tude Pratique : \\
Site Web pour conf�rence scientifique} }
\markright{Site Web de conf�rence scientifique} 
                           % Pour avoir le titre de l'expose sur chaque page

\author{Quentin \textsc{Dufour}, Thomas \textsc{Hareau}, \\
        Laurent \textsc{Aymard}, Jean \textsc{Chorin} \\
        \\
        Encadreur : Jean-Fran�ois \textsc{Dupuy}}

\date{2015}                    % Ne pas modifier
 
%%%%%%%%%%%%%%%%%%%%%%%%%%%%%%%%%%%%%%%%%%%%%%%%%%%%%%%%%%%%%%%%%%%%%%%%%%%%%%%

\begin{document}          

\maketitle                 % G�n�re le titre
\thispagestyle{empty}      % Supprime le num�ro de page sur la 1re page
\newpage

\tableofcontents
\newpage

\begin{abstract}
%Resum�
\end{abstract} 
\newpage

\section{Introduction}  


\section{Titre de section}  

Ligne de remplissage pour visualiser la mise en page. Ligne de
remplissage pour visualiser la mise en page. 

\subsection{Titre de sous-section}


\subsubsection{Titre de sous-sous-section}


\paragraph{Titre de paragraphe}


\subsection{Encore un titre de sous-section}

Exemple de liste � puces :
\begin{itemize}
        \item ligne de remplissage pour visualiser la mise en page. Ligne de
        remplissage pour visualiser la mise en page ;

        \item ligne de remplissage pour visualiser la mise en page. Ligne de
        remplissage pour visualiser la mise en page.
\end{itemize}

Ligne de remplissage pour visualiser la mise en page. Ligne de remplissage pour
visualiser la mise en page. 


\section{Conclusion} 
 
\LaTeX\ c'est facile pour produire des documents standard et nickel ! 
Et Bib\TeX\ pour les r�f�rences, c'est le pied.

\bibliography{biblio}


\end{document}

%%%%%%%%%%%%%%%%%%%%%%%%%%%%%%%%%%%%%%%%%%%%%%%%%%%%%%%%%%%%%%%%%%%%%%%%%%%%%%%
