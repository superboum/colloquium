%%%%%%%%%%%%%%%%%%%%%%%%%%%%%%%%%%%%%%%%%%%%%%%%%%%%%%%%%%%%%%%%%%%%%%%%%%%%%%%
%     STYLE POUR LES EXPOS�S TECHNIQUES 
%         3e ann�e INSA de Rennes
%
%             NE PAS MODIFIER
%%%%%%%%%%%%%%%%%%%%%%%%%%%%%%%%%%%%%%%%%%%%%%%%%%%%%%%%%%%%%%%%%%%%%%%%%%%%%%%

\documentclass[a4paper,11pt]{article}

\usepackage{exptech}       % Fichier (./exptech.sty) contenant les styles pour 
                           % l'expose technique (ne pas le modifier)

%\linespread{1,6}          % Pour une version destin�e � un relecteur,
                           % d�commenter cette commande (double interligne) 
                           
% UTILISEZ SPELL (correcteur orthographique) � acc�s simplifi� depuis XEmacs

%%%%%%%%%%%%%%%%%%%%%%%%%%%%%%%%%%%%%%%%%%%%%%%%%%%%%%%%%%%%%%%%%%%%%%%%%%%%%%%

\title{ \textbf{Site Web pour conf�rence scientifique\\
Documentation Technique} }
\markright{Site Web de conf�rence scientifique - Documentation Technique} 
                           % Pour avoir le titre de l'expose sur chaque page

\author{Quentin \textsc{Dufour}, Thomas \textsc{Hareau}, \\
        Laurent \textsc{Aymard}, Jean \textsc{Chorin} \\
        \\
        Encadreur : Jean-Fran�ois \textsc{Dupuy}}

\date{2015}                    % Ne pas modifier
 
%%%%%%%%%%%%%%%%%%%%%%%%%%%%%%%%%%%%%%%%%%%%%%%%%%%%%%%%%%%%%%%%%%%%%%%%%%%%%%%

\begin{document}          

\maketitle                 % G�n�re le titre
\thispagestyle{empty}      % Supprime le num�ro de page sur la 1re page
\newpage

\tableofcontents
\newpage

\begin{abstract}
%Resum�
\end{abstract} 
\newpage

\section{Introduction}  

Lorem Ipsum
\newpage

\section{Trucs tous b�tes et tr�s utiles}

Une liste de choses � savoir d�s le d�but pour prendre en main
\newpage

\section{Logiciels/frameorks utilis�s}

La liste
%Comment on les utilise
\newpage

\subsection{HTML/CSS}

\begin{itemize}
\item Bootstrap
\end{itemize}
\newpage

\subsection{Serveur}

\begin{itemize}
\item Sinatra
\item Rake
\item minitest
\end{itemize}
\newpage

\subsection{Base de donn�es}

\begin{itemize}
\item sqlite
\item Active-record
\end{itemize}
\newpage


\section{Structure de la Base de Donn�e}  

Lorem ipsum

(Avec un b� diagramme UML)
\newpage


\section{Sch�ma de Navigation} 

Arborescence des fichiers
Leur r�le
\newpage


\section{Interface administration} 

Fonctionnalit�s
Ce qu'elle fait
\newpage


\section{Conclusion} 
 
Lorem Ipsum
\newpage


\bibliography{biblio}


\end{document}

%%%%%%%%%%%%%%%%%%%%%%%%%%%%%%%%%%%%%%%%%%%%%%%%%%%%%%%%%%%%%%%%%%%%%%%%%%%%%%%
